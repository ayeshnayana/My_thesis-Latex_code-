

\begin{center}

\textbf{AUTOBIOGRAPHICAL STATEMENT}
\end{center}
\noindent
\textbf{Name}:
\noindent
Ayesh Gunawardana
\singlespacing

\noindent
\textbf{Education}:


\noindent
B.S. Physics, University of Kelaniya, Kelaniya, Sri Lanka, 2013

\noindent
\textbf{Professional Experience}:

\noindent
Rumble Fellow, Dept. of Physics and Astronomy, Wayne State University, 2018-2019\\
Graduate Research Assistant, Dept. of Physics and Astronomy, Wayne State University, 2017-2018\\
Graduate Teaching Assistant, Dept. of Physics and Astronomy, Wayne State University, 2014-2017\\
Teaching Assistant, Dept. of Physics, University of Kelaniya, Sri Lanka, 2013-2014\\

\noindent
\textbf{Publications}: 
\begin{itemize}
\item C. M. Grant, A. Gunawardana, and A. A. Petrov, \textit{Semileptonic decays of heavy mesons with artificial neural networks}, [arXiv:1912.09058 [hep-ph]]
\item A. Gunawardana, \textit{Reevaluating uncertainties in $\bar{B}\to X_s\gamma$ decay} [arXiv:1909.09081] (DPF 2019 Conference proceedings)
\item 	A. Gunawardana and G. Paz, \textit{Reevaluating uncertainties in $\bar{B}\to X_s\gamma$ decay}, JHEP 11, 141 (2019) [arXiv:1908.02812] 
\item	A. Gunawardana and G. Paz, \textit{On HQET and NRQCD operators of dimension 8 and above}, JHEP 1707, 137 (2017), [arXiv:1702.08904] 
\end{itemize}

\singlespacing
