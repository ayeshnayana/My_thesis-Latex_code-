\chapter{Conclusion and future work}
\section{Conclusion}
Flavor physics is the study of different species of elementary particles \cite{flavor_phys}, and it provides tools to expand the boundaries of SM. For example, the radiative FCNC decay $\bar{B}\to X_s\gamma$ is considered as one of the standard candles of BSM \cite{Neubert:2005mu}. As another example, semileptonic decays of heavy mesons provide the means to extract the CKM matrix elements. Due to the effects of QCD, these decays are plagued with nonperturbative uncertainties. Therefore, it is important to control these uncertainties to the understand new physics in above processes. 
\par 
In this work we discussed the controlling the nonperturbative uncertainties in $\bar{B}\to X_s\gamma$ and semileptonic $D\to \pi l v$. In chapter \ref{chap:Matrix_elements}, we provided the construction of a new basis for the tensor decomposition of HQET and NRQCD matrix elements of any dimension. In chapter \ref{reevaluating_b_to_x}, we used the new basis for HQET/NRQCD to obtain the higher dimensional moments of subleading shape function. This function parameterizes the nonperturbative effects. We used these moments to model the subleading shape function and reevaluate the nonperturbative uncertainties. Finally, in chapter \ref{ANN_paper}, we used artificial neural networks to parameterize the shape of the form factor $F_+(q^2)$, which describes the nonperturbative effects of $D\to\pi l v$. 

\subsection{On HQET and NRQCD operators of dimension eight and above}
In chapter \ref{chap:Matrix_elements}, we provided the method to construct operators for the HQET and NRQCD Lagrangians at any given dimension. Although these theories employ different power counting schemes, the Lagrangians are closely related \cite{Manohar:1997qy}. We analyzed operators that contain two HQET fields or two NRQCD (NRQED) fields with an arbitrary number of covariant derivatives. These matrix elements can be written as nonperturbative HQET parameters multiplied by tensors constructed from the heavy quark velocity, the metric tensor, and the Levi-Civita tensor. We also use constraints
coming from the time-reversal (T) and parity (P) symmetries, hermitian conjugation, and the fact that we work in 3 + 1 dimension. At a given dimension, the number of allowed HQET operators are equivalent to the number of HQET parameters up to a possible color factor.\par
The new basis of HQET matrix elements allows us to easily determine the number of allowed HQET/NRQCD operators at a given dimension. This decomposition of matrix elements allows us to check whether given operators are linearly independent. The method also allows relating operators to one another easily. Following this, we constructed the HQET and
NRQCD Lagrangian at mass dimension 8 for the first time.\par
As shown in \cite{Kobach:2017xkw}, operators that contain symmetric product of two color matrices, such as  $\psi^\dagger E^i_aT^a  E^i_bT^b\psi$, can be decomposed in terms of a color octet and a color singlet operators, $\psi^\dagger E^i_a E^i_b\,d^{abc}T^c \psi$ and $\psi^\dagger E^i_a E^i_b \delta^{ab}\psi$. Since they only differ in their color structure, both will give the same linear combination of parameters.  Alternatively we can use the basis of  $\psi^\dagger E^i_a E^i_b\left\{T^a,T^b\right\}\psi$ and $\psi^\dagger E^i_a E^i_b \delta^{ab}\psi$. The operator $\psi^\dagger E^i_a E^i_b\left\{T^a,T^b\right\} \psi$ is generated by commutator and anti-commutators of covariant derivatives, and it is the only of the two that appears when calculating observables at tree level.  The operator $\psi^\dagger E^i_a E^i_b \delta^{ab}\psi$ will be generated when considering radiative corrections \cite{Manohar:1997qy}. For applications to inclusive $B$ decays, this operator arises only at order $\alpha_s/m^4_b$, beyond the current level of precision \cite{Gunawardana:2017zix}. Using the method presented above allows determining how many linearly independent operators there are for possible different color structures.\par
In section \ref{relating_with_lit}, we relate the HQET parameters of operators of dimension four, five, six, seven, and eight known from the literature to our basis. NRQCD operators up to dimension seven and NRQED operators up to dimension eight were previously known in the literature. We related these operators to the corresponding HQET matrix elements. The relation between the HQET/NRQCD operators and the matrix elements allows us to write the operators in terms of nonperturbative HQET parameters.\cite{Gunawardana:2017zix}.\par
In section \ref{applications}, we analyzed dimension nine spin-independent HQET parameters. Here we found 24 possible parameters (not including multiple color structures). Most importantly, we constructed the dimension eight NRQCD operators that do not appear in the $1/M^4$ NRQED Lagrangian. These allow presenting for the full $1/M^4$ bilinear NRQCD Lagrangian.\par
\subsection{Reevaluating the uncertainties in $\bar{B}\to X_s\gamma$ }
The section \ref{sec:chap5_moments_g_17} provides the moments of subleading shape function using data given in \cite{Gambino:2016jkc} and the basis developed in chapter \ref{chap:Matrix_elements} \cite{Gunawardana:2017zix}. This subleading shape function relates to the soft function $h_{17}$, which parameterize the nonperturbative uncertainty. The function $h_{17}$ defined in equation (\ref{eqn:chapetr3_h_17}), and it has the following properties: it is a real and even function over gluon momentum $\omega_1$, it's odd moments over $\omega_1$ vanish and it has dimensions of mass. Based on these properties we developed a new model for soft function $h_{17}$ based on a combination of \textit{Hermite polynomials} multiplied \textit{Gaussian}. The explicit form of the new model is given in section \ref{Model}.\par
The $h_{17}$ is is a soft function, so one expects it not to have significant structures beyond $\omega_1\leq 1$ GeV. This provides other constraints such as $\left|h_{17}\left(\omega_{1}\right)\right| \leq 1\mathrm{\, GeV}$, and it limits the function from having structures beyond $|\omega_1|\leq 1\text{ GeV}$. Scanning through different values of moments, we found a new estimate for $Q_1^q-Q_{7\gamma}$. Our estimate reduced the 2010 estimate \cite{Benzke:2010js} by a third. Also, we combined the new estimate for $Q_{7\gamma}-Q_{8g}$ with our new result for $Q_1^q-Q_{7\gamma}$ to obtain a new range for the total rate. Following from this we found that the uncertainty of the total rate is reduced by half compared to the 2010 values \cite{Benzke:2010js}.\par
The SM prediction for CP asymmetry is obtained by nonperturbative parameters $\tilde{\Lambda}_{17}^u$ and $\tilde{\Lambda}_{17}^c$. These parameters are also related to $h_{17}$. We reevaluated their ranges using our analysis. From this we found a new estimate for SM CP asymmetry as $-1.9 \%<\mathcal{A}_{X_{s} \gamma}^{\mathrm{SM}}<3.3 \%$, which is an increased range compared to the 2010 estimate \cite{Benzke:2010tq}. This is because of the increased range of the $\tilde{\Lambda}_{17}^u$.\par

\subsection{Semileptonic decays of heavy mesons with artificial neural networks}
The CKM matrix element $|V_{cd}|$ can be extracted most easily from semileptonic $D\rightarrow\pi l\nu$ decays. This decay is parameterized by the form factor $F_{+}(q^2)$, which describes the hadronic part of the decay amplitude. Experimentally, this form factor is studied by analyzing the differential decay rate of $d\Gamma/dq^2$. The lack of precise information about these form factors from the first principles of QCD is one of the main sources of uncertainties when extracting the CKM parameters. In spite of the recent progress from lattice QCD (LQCD), still, there is no ab-initio approach to describe the shape of the form factors in the whole physical region of momentum transfer $q^2$.\par
In view of the lack of first principle calculation for decay rates,  phenomenological parameterizations of form factors are used to model the shape. These models first estimate the hadronic form factor at one kinematic point and then extrapolate based on the assumed functional shape of the form factor. What systematic uncertainty does choosing a particular function brings to such extrapolation? To answer this question, we use machine learning (ML) framework. In section \ref{ANN}, we used artificial neural networks (ANN) as an unbiased estimator of data.\par 
We used an average of 100 feed-forward ANN with two hidden layers to obtain the unbiased estimate for $|V_{cd}|F_{+}(q^2)$. Each hidden layer contains 100 nodes \cite{Grant:2019yar}. The ANN performance was improved by using NLCG optimization method. We observed that NLCG method is 200 times faster compared to the common optimization methods such as gradient descent. The python code for ANN, results of the ANN training and relevant graphs are available at \url{https://s.wayne.edu/hepmachinelearning/}. The comparison between our ANN extrapolation of $|V_{cd}|F_{+}(q^2)$ against the existing models is provided in figure \ref{fig:VFwithModels}. From this we observe ANN fit is consistent with all the phenomenological models at low $q^2$ region. This implies that the ANN successfully extrapolated the $|V_{cd}|F_{+}(q^2)$ data to $q^2\to 0$.
 Most importantly, our ANN fit provides model independent parameterization of the $F_+(q^2)$ shape for the first time in literature. This was instrumental in developing the unitarity constraints on the form factor, which allowed for model-independent bounds on $V_{cd}$ \cite{Grant:2019yar}. 
\section{Future work}
We conclude with a remark on future developments. In section \ref{sec:gen_method}, we provided a general method of writing down all the possible HQET operators of any given dimension. It would be interesting to automatize the procedure using a computer program to construct these higher dimensional operators and the NRQCD Lagrangian. Also, certain multiple color structures were considered separately from the general method. It would be desirable to find a method that automatically generates these color structures.\par
We have not considered operators with more than two HQET or NRQCD (NRQED) fields. The one non-relativistic fermion sector can be combined with an additional non-relativistic field or an additional relativistic field.  Results for each case were presented in the literature  \cite{Bodwin:1994jh, Brambilla:2006ph, Brambilla:2008zg, Hill:2012rh, Dye:2016uep}, but not for an arbitrary operator dimension.\par
With the new information on the moments, we can better control the hadronic effects. However, the scale dependence on $1/m_b$ corrections is not fully controlled because, currently, we treat them at the leading order in $\alpha_s$. Therefore, to improve the $Q_{1}^q-Q_{7\gamma}$ contributions further, we need to take account of the $\alpha_s$ corrections.\par
Our model relies on the numerical estimates of the matrix elements of dimension eight operators. Still, it could further be improved if we knew the numerical estimates of dimension nine matrix elements. With the Belle II data, we can hope to have improvements on this.\par
In section \ref{sec:chap5_moments_g_17}, we only considered the quantities that are integrated over photon energy. The above moment information can be used to model $Q_{1}^q-Q_{7\gamma}$ contributions for quantities that are not integrated over photon spectrum.\par
The large amount of data that will be obtained at the B factory
Belle II will be ideal for deep learning. The extraction of CKM parameters using machine learning can be further improved with the new knowledge of optimization methods and deep learning. For example, our work provided in section \ref{ANN} uses the nonlinear conjugate methods (NLCG) to minimize the error function in the training process. The NLCG is a local optimizer. Whereas stochastic methods such as simulated annealing could have been more successful in optimizing the error function, and they are more suitable for deep learning. This is because these stochastic methods provide global optimization. With a large data set, we expect to develop deep neural networks utilizing parallel computation and graphical processing units.


%