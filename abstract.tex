
\begin{center}
\textbf{ABSTRACT}
	
	
	\singlespacing
\textbf{EFFECTIVE FIELD THEORY AND MACHINE LEARNING APPROACHES TO CONTROLLING NONPERTURBATIVE UNCERTAINTIES IN FLAVOR PHYSICS}\\
	\doublespacing
	
	by\\
	
	\textbf{AYESH GUNAWARDANA}\\
	\textbf{August 2020} \\
\end{center}
\begin{tabular}{ll}	
\textbf{Advisor:} &Dr. Gil Paz \\
\textbf{Major:}   &Physics\\
\textbf{Degree:}  &Doctor of Philosophy\\
\end{tabular}
\bigskip
\par
The radiative decay $\bar{B}\to X_s\gamma$ and semileptonic heavy meson decay $D\to \pi l \nu$ are important flavor physics probes of new physics. However, these decays are
plagued with nonperturbative uncertainties that are needed to be controlled to
obtain a theoretically clean description. In this dissertation, we provide effective
field theory and machine learning approaches to controlling these uncertainties\par
In $\bar B\to X_s\gamma$, the largest uncertainty on the total rate arises from $Q_1-Q_{7\gamma}$ operator pair.  This contribution is given by a soft function whose moments are
related to nonperturbative heavy quark effective theory (HQET) operators’ matrix elements. The extraction of higher-order moments requires the knowledge of
higher dimensional HQET operators. We present a general method that allows
for an easy construction of HQET and non-relativistic quantum-chromo dynamics
(NRQCD) operators containing \textit{any} number of covariant derivatives. As an application, we list, for the first time, all operators in the dimension eight NRQCD
Lagrangian. Then we use recently extracted HQET matrix elements to reevaluate
the nonperturbative uncertainty of $\bar B\to X_s\gamma$  total decay rate and CP asymmetry.\par
The decay rate of semileptonic $D\to \pi l \nu$  is proportional to the hadronic form factors. Currently, these form factors cannot be determined analytically in the
whole range of available momentum transfer $q^2$, but can be parameterized with
a varying degree of model dependency. We propose a machine learning approach
with artificial neural networks trained from experimental pseudo-data to predict
the shape of these form factors with a prescribed uncertainty. This provides the
first model-independent parameterization of $D\to \pi l \nu$ vector form factor shape in the literature. 

 