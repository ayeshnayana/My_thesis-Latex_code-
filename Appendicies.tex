\appendix
\chapter{}
\section{Running of Wilson coefficients $C_1, C_7$ and $C_{8g}$}\label{app:Wilson}

The expressions for the running of Wilson coefficients $C_1, C_7$ and $C_{8g}$ are found in \cite{Buras:1998raa, Manohar:2000dt}.
\begin{eqnarray}
\begin{aligned} C_{j}^{(0)}\left(\mu_{b}\right) &=\sum_{i=1}^{8} k_{j i} \eta^{a_{i}} \quad(j=1, \ldots, 6) \\ C_{7 \gamma}^{(0) e f f}\left(\mu_{b}\right) &=\eta^{\frac{16}{23}} C_{7 \gamma}^{(0)}\left(\mu_{W}\right)+\frac{8}{3}\left(\eta^{\frac{14}{23}}-\eta^{\frac{16}{23}}\right) C_{8 G}^{(0)}\left(\mu_{W}\right)+C_{1}^{(0)}\left(\mu_{W}\right) \sum_{i=1}^{8} h_{i} \eta^{a_{i}} \\ C_{8 G}^{(0) e f f}\left(\mu_{b}\right) &=\eta^{\frac{14}{23}} C_{8 G}^{(0)}\left(\mu_{W}\right)+C_{1}^{(0)}\left(\mu_{W}\right) \sum_{i=1}^{8} \overline{h}_{i} \eta^{a_{i}} \end{aligned}
\end{eqnarray}
where
\begin{eqnarray}
\eta=\frac{\alpha_{s}\left(\mu_{W}\right)}{\alpha_{s}\left(\mu_{b}\right)}
\end{eqnarray}
To evaluate the Wilson coefficients at $\mu=1.5\text{ GeV}$ we use following boundary conditions \cite{Mohr:2015ccw, Tanabashi:2018oca}.
\begin{eqnarray}
\begin{aligned}
m_Z=(91.188\pm 0.002)\mbox{ GeV}\\
m_W=(80.379\pm 0.001)\mbox{ GeV}\\
\alpha_s(M_Z)=0.1181(1)\\
m_t^{\text{pole}}=172.9\pm0.4\mbox{ GeV}\\
m_b=4.18^{0.03}_{-0.02}\mbox{ GeV}
\end{aligned}
\end{eqnarray}
Note that the quark masses are given in pole mass scheme. However, pole masses of quarks can be converted into $\overline{\text{MS}}$ scheme as follows \cite{Tanabashi:2018oca}:
\begin{eqnarray}\label{polemss}
\overline{m}_{q}=m^{pole}_{q}\left(m^{pole}_{q}\right)\left(1-\frac{4 \alpha_{s}\left(m^{pole}_{q}\right)}{3 \pi}+\mathcal{O}\left(\alpha_{s}^{2}\right)\right)
\end{eqnarray}
The top quark mass obtained in equation (\ref{polemss}) is scaled down to $\mu=M_W$. Therefore, this gives us $m_t^{\overline{\text{MS}}}(\mu=M_W)=175.05\text{ GeV}$.\\
In addition, the $C_{7 \gamma}^{(0)}, C_{8 G}^{(0)}$ and $C_{1}^{(0)}$ are given by:
\begin{eqnarray}
\begin{array}{c}{C_{1}^{(0)}\left(\mu_{W}\right)=1} \\ {C_{7 \gamma}^{(0)}\left(\mu_{W}\right)=\frac{3 x_{t}^{3}-2 x_{t}^{2}}{4\left(x_{t}-1\right)^{4}} \ln x_{t}+\frac{-8 x_{t}^{3}-5 x_{t}^{2}+7 x_{t}}{24\left(x_{t}-1\right)^{3}} \equiv-\frac{1}{2} D_{0}^{\prime}\left(x_{t}\right)} \\ {C_{8 G}^{(0)}\left(\mu_{W}\right)=\frac{-3 x_{t}^{2}}{4\left(x_{t}-1\right)^{4}} \ln x_{t}+\frac{-x_{t}^{3}+5 x_{t}^{2}+2 x_{t}}{8\left(x_{t}-1\right)^{3}} \equiv-\frac{1}{2} E_{0}^{\prime}\left(x_{t}\right)}\end{array}
\end{eqnarray}
where $x_t=\frac{m_t^2}{M_W^2}$
Also, running of the $C_1$ is given by 
\begin{eqnarray}\label{c1new}
C_{1}^{(0)}(\mu)=\frac{1}{2} \eta^{-12 / 23}+\frac{1}{2} \eta^{6 / 23}
\end{eqnarray} 
\section{Appendix B : Useful identity}\label{appen_b}
The  Wilson line 
\begin{equation}
S_{\bar n}(x)=\mbox{\bf P}\exp \left( ig\int_{-\infty}^0 \,du \,\bar n\cdot A_s(x+u\bar n) \right)\,,
\end{equation}
obeys the equation $i\bar n\cdot D\,S_{\bar n}(x)=0$, where $iD^\mu=i\partial^\mu+gA^\mu$, see, e.g., \cite{Becher:2014oda} for a derivation.  Thus $i\bar n\cdot\partial S_{\bar n}(x)=-g\bar n\cdot A(x)S_{\bar n}(x)$. Taking the Hermitian conjugate of this identity gives $i\bar n\cdot\partial S^\dagger_{\bar n}(x)=S^\dagger_{\bar n}(x)g\bar n\cdot A(x)$. Consider now $i\bar n\cdot \partial \left(S_{\bar n}^\dagger(x) O(x) S_{\bar n}(x) \right)$, where $O(x)$ is an operator. Using the identities above we have 
\begin{eqnarray}
&&i\bar n\cdot \partial \left(S_{\bar n}^\dagger(x) O(x) S_{\bar n}(x) \right)=\nonumber\\
&=&\big(i\bar n\cdot \partial S_{\bar n}^\dagger(x)\big)O(x) S_{\bar n}(x) +S_{\bar n}^\dagger(x)\big(i\bar n\cdot \partial\, O(x)\big)S_{\bar n}(x) +S_{\bar n}^\dagger(x)O(x) \big(i\bar n\cdot \partial S_{\bar n}(x) \big)\nonumber\\
&=&S^\dagger_{\bar n}(x)g\bar n\cdot A(x)O(x) S_{\bar n}(x) +S_{\bar n}^\dagger(x)\big(i\bar n\cdot \partial\, O(x)\big)S_{\bar n}(x)-S_{\bar n}^\dagger(x) O(x) g\bar n\cdot A(x)S_{\bar n}(x)=\nonumber\\
&=& S^\dagger_{\bar n}(x)[g\bar n\cdot A(x),O(x)] S_{\bar n}(x)+S_{\bar n}^\dagger(x)\big[i\bar n\cdot \partial ,O(x)\big]S_{\bar n}(x)=S_{\bar n}^\dagger(x)\big[i\bar n\cdot D,O(x)\big]S_{\bar n}(x).
\end{eqnarray}
In the last line we have used the identity $\big[i\bar n\cdot \partial ,O(x)\big]f(x)=\big(i\bar n\cdot \partial\, O(x)\big)f(x)$ for an arbitrary function $f(x)$. Thus we have the identity 
 \begin{equation}
 i\bar n\cdot \partial \left(S_{\bar n}^\dagger(x) O(x) S_{\bar n}(x) \right)=S_{\bar n}^\dagger(x)\big[i\bar n\cdot D,O(x)\big]S_{\bar n}(x).
 \end{equation}